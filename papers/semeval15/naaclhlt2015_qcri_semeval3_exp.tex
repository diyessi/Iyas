\section{Submissions and Results}
\label{sec:experiments}

Now we describe our primary submissions to the three tasks, followed by the 
contrastive submissions. Table~\ref{tab:results} includes our official 
competition results; all the reported $F_1$ values are macro-averaged.

\subsection{Primary Submissions}

In general, our approaches perform multi-class classification on the basis of a 
one-vs-rest support vector machines strategy (\ie we train one classifier for 
each class). Our classifications for both Arabic and English A are at the
comment level.

\begin{table}%[h]
\centering
\footnotesize
%  \begin{tabular}{|l|c@{\hskip 0.2cm}c@{\hskip 0.2cm}c@{\hskip 0.2cm}c|}
\begin{tabular}{|l|cccc|}
  \hline
  \bf ar	& \dir & \texttt{IRREL} & \rel & \texttt{F$_1$}\\  \hline  
  primary	& $77.31$ & $91.21$	& $67.13$	&  $78.55$ \\
  cont$_1$	& $74.89$ & $91.23$	& $63.68$	&  $76.60$ \\
  cont$_2$	& $76.63$ & $90.30$	& $63.98$	& $76.97$ \\  
%   primary	& $77.31$ & $91.21$	& $67.13$	& $78.55$ \\
%   cont$_1$	& $74.89$ & $91.23$	& $63.68$	& $76.60$ \\
%   cont$_2$	& $76.63$ & $90.30$	& $63.98$	& $76.97$ \\
  \hline \hline

  \bf en A	& \good   & \bad 	& \texttt{POT}	& \texttt{F$_1$}\\\hline
  primary	& $78.45$ & $72.39$	& $10.40$	& $53.74$ \\
  cont$_1$ 	& $76.08$ & $75.68$	& $17.44$	& $56.40$ \\
  cont$_2$ 	& $75.46$ & $72.48$ 	& $\,\,\,7.97$	& $51.97$ \\
\hline  \hline

\bf en B	& \yes	  & \no		& \unsure	& F$_1$	 \\
  \hline  
  primary	& $80.00$ & $44.44$	& $36.36$	& $53.60$ \\
  cont$_1$ 	& $75.68$ & $\,\,\,0.00$& $\,\,\,0.00$	& $25.23$ \\
  cont$_2$ 	& $66.67$ & $33.33$ 	& $47.06$	& $49.02$ \\
  \hline
 \end{tabular}
\caption{Per-class and macro-averaged $F_1$-measure of our primary and 
contrastive submissions to SemEval Task 3 for Arabic (ar) and English 
(en) A and B.
\label{tab:results}}
\end{table}

\paragraph{Arabic} 
Our submission applies the logistic regressor from scikit-learn.%
\footnote{http://scikit-learn.org/stable/}
The features are lexical similarities (Section~\ref{sub:sim}) and  $n$-grams 
(Section~\ref{ssub:ngrams}). In a sort of stacking, the output of our 
contrastive submission~1 is included as another feature (\cf 
Section~\ref{sub:contrastive}). 

This submission achieved the first position in the competition (F$_1=78.55$, 
compared to $70.99$ for the second one). It showed a particularly high 
performance when labeling \rel comments.

\paragraph{English A}
The submission applies a linear-kernel SVM from scikit-learn. We used a 
one-versus-rest approach to account for the fact that the learning problem is a 
multiclass one. We tuned the value of the $C$ hyper-parameter of the SVM in 
order to deal with class imbalance ---by increasing the value of $C$, we built 
more complex classifiers for those classes with less instances. The features for 
this submission consist of lexical, syntactic, and semantic similarities 
(Section~\ref{sub:sim}), context information (Section~\ref{ssub:context}), 
$n$-grams (Section~\ref{ssub:ngrams}), and heuristics 
(Section~\ref{ssub:heuristics}). Similarly to the Arabic submission, the output 
of our rule-based system from the contrastive submission~2 is another feature. 

This submission achieved the third position in the competition (F$_1=53.74$, 
compared to $57.19$ for the top one). \pot comments showed to be the 
hardest ones to identify, as the border with respect to the rest of the comments 
is very fuzzy. Indeed, a manual inspection on some random comments show that the 
decision between \good and \pot comments is nearly impossible in some cases.

\paragraph{English B}

Following the manual labeling strategy applied to the \yes/\no questions by the 
task organizers~\cite{Marquez-EtAl:2015:SemEval}, our approach consists of three 
steps:
\Ni identifying the \good comments among those associated with $q$;
\Nii classifying each of them as \yes, \no, or \unsure; and 
\Niii aggregating the comment-level classifications into question-level. The 
overall answer to $q$ becomes that of the majority of the comments. In case of a
draw, we opt for labeling it as \unsure.%
\footnote{The majority class in the training and dev.\ sets (\yes), could have 
been the default answer. Still, we opted for a conservative decision: deciding 
\unsure if no evidence enough was at hand.}
Step \Ni is indeed task A. As for step \Nii, we train a classifier as that for 
English task A, but adding the polarity and user's profile features (\cf 
Sections~\ref{sub:polarity} and~\ref{sub:profile}).%
\footnote{Even if the user's profile information seems to fit with task A, 
rather than B, at development time they showed to be effective only for the 
latter one.}

This submission achieved the third position in the competition (F$_1=53.60$, 
compared to $63.70$ for the top one). Differently to the rest of the tasks, our 
submitted results were obtained with a classifier trained on the training data 
only (the development set was neglected). The reason behind this decision was 
that we obtained an unexpected distribution of mostly \yes predictions on the 
test set when both training and development sets had been considered. Such 
distribution is completely different to that observed in both training and 
development partitions. Further experiments, carried out after the submission, 
demonstrated that the causes for such an unexpected behavior were bugs in the 
implementation of some features and the fact that some features were computed on 
unreliable statistics of the data. Further discussion is included in 
Section~\ref{sec:discussionb}.


\subsection{Contrastive Submissions}
\label{sub:contrastive}

\paragraph{Arabic} 

We approach our contrastive submission~1 as a ranking problem. After 
stopwording and stemming, $sim(q,c)$ is computed as 
\begin{equation}
 sim(q,c) = \frac{1}{|q|} \sum_{t\in q\cap c} \omega(t) \enspace ,
 \label{eq:overlap}
\end{equation}
% 
where the empirically-set weight $\omega(t)=1$ if $t$ is a $1$-gram and 
$\omega(t)=4$ if $t$ is a $2$-gram. Given the 5 comments $c_1,\ldots,c_5\in C$ 
associated to $q$, the maximum similarity $\max_C sim(q,c)$ is mapped to a 
maximum 100\% similarity and the rest of the scores are mapped proportionally. 
Each comment is assigned a class according to the following ranges: [80, 100]\% 
for \dir, (20,80)\% for \rel, and [0,20]\% for \irel. Threshold values manually 
tuned on the training data.


As for the contrastive submission~2, we built a binary classifier \dir vs. 
\texttt{NO-}\dir based on logistic regression. The comments are then sorted 
according to the classifier's prediction confidence and the final labels are 
assigned accordingly: \dir for the 1st ranked, \rel for the 2nd ranked, and 
\irel for the rest. Only lexical similarities are included as features 
(discarding those weighted with idf variants).


The performance of these two submissions is below but close to that of the 
primary one (F$_1=76.60$ and $76.97$), particularly when identifying \irel 
comments. 

\paragraph{English A}

For our contrastive submission~1, the same machine learning schema as for the 
primary submission is used, but now using 
SVM$^\mathrm{light}$~\cite{Joachims:99}. This toolkit allows us to deal with 
the class imbalance by tuning the $j$ parameter (cost of making mistakes on 
positive examples). This time the $C$ hyper-parameter is set to the default 
value. As we focused on improving the performance on \pot instances, we 
obtained better results for this category (F$_1=17.44$), surpassing the overall 
performance from the primary submission (F$_1=56.40$).

Our contrastive submission~2 operates in the same way as the Arabic contrastive 
submission~1. The applied ranges are the same, but this time for \good, \pot, 
and \bad. Some heuristics override the so generated decisions: $c$ is classified 
as \good if it includes a URL, starts with an imperative verb (\eg \textit{try}, 
\textit{view}, \textit{contact}, \textit{check}), or contains \textit{yes words} 
(\eg \textit{yes}, \textit{yep}, \textit{yup}) or \textit{no words} (\eg 
\textit{no}, \textit{nooo}, \textit{nope}). Comments written by the author of 
the question or including acknowledgments are considered dialogues and 
classified as \bad. The result of this submission is slightly lower than the 
others' (F$_1=51.97$), where the automatic learning allows for better 
predictions.

\paragraph{English B}

Our contrastive submission~1 is identical to our primary one, but it uses both 
the training and the development data for training the model. The reason behind 
the disastrous results (F$_1=25.23$) is a buggy implementation of some of the 
polarity features (\cf Section~\ref{sub:polarity}) and the lack of statistics 
for properly estimating category-level user profiles (\cf 
Section~\ref{sub:profile}). 
 
 
The contrastive submission~2 consists of a rule-based system. A comment is 
labeled as \yes if it starts with affirmative words: \textit{yes}, 
\textit{yep}, \textit{yeah}, etc. It is labeled as \no if it starts with 
negative words: \textit{no}, \textit{nop}, \textit{nope}, etc. The answer to $q$ 
becomes that of the majority of the comments ---\unsure in case of tie. It is 
worth noting the comparably high performance when dealing with \unsure questions 
(F$_1=47.06$) with this simple rationale.