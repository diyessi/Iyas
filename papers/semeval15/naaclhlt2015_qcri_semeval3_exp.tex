\section{Submissions and Results}
\label{sec:experiments}

Now we describe our primary submissions to the three tasks, followed by the 
contrastive submissions. Table~\ref{tab:results} includes our official 
competition results.

% \blue{Massimo/Simone: please, give more details about this, if necessary}


\begin{table}%[h]
 \begin{tabular}{|l|c@{\hskip 0.2cm}c@{\hskip 0.2cm}c@{\hskip 0.2cm}c|}
%   \multicolumn{5}{c}{Subtask A} \\
%   \hline
%  Subm.		& \bf \good & \bf\bad 	& \bf \texttt{POT} & 	
%  \\
  \hline
  \bf ar& \bf\dir & \bf\rel &\bf \texttt{IRREL} & \bf\texttt{MACRO}\\  \hline  
  prim	& $77.31$ & $91.21$	& $67.13$	& $78.55$ \\
  cont$_1$	& $74.89$ & $91.23$	& $63.68$	& $76.60$ \\
  cont$_2$	& $76.63$ & $90.30$	& $63.98$	& $76.97$ \\
    \hline \hline
 

  \bf en A & \bf \good & \bf\bad & \bf \texttt{POT} & \bf\texttt{MACRO} \\\hline
  prim  & $78.45$	& $72.39$	& $10.40$	& $53.74$ \\
  cont$_1$ & $76.08$	& $75.68$	& $17.44$	& $56.40$ \\
  cont$_2$ & $75.46$	& $72.48$ 	& $7.97$	& $51.97$ \\
% \end{tabular}
% \begin{tabular}{|l|crrr|}
% \hline
%   \multicolumn{5}{c}{Subtask B} \\
\hline  \hline
 \bf en B	& \bf \yes & \bf \no & \bf \unsure & \bf \texttt{MACRO}	 \\
  \hline
  
  prim	& $80.00$	& $44.44$	& $36.36$	& $53.60$ \\
  cont$_1$& $75.68$	& $0.00$	& $0.00$ & $25.23$ \\
  cont$_2$ & $66.67$	& $33.33$ 	& $47.06$	& $49.02$ \\
  \hline
 \end{tabular}
\caption{Per-class and overall $F_1$-measure of our \textit{prim}ary and 
\textit{cont}rastive submissions to SemEval Task 3 for Arabic (A) and English 
(en).
% The columns 2-4 specify the F1 measure of 
% the binary learning problem, for example Good VS all other classes. Finally 
the 
% 6-th column shows the official F1 score for the submission.
\label{tab:results}}
\end{table}

\subsection{Primary Submissions}

In general, our approaches perform multi-class classification on the basis of a 
one-vs-rest support vector machines strategy (\ie we train one classifier for 
each class). Our classifications for both Arabic and English A are made at 
comment level.


\paragraph{Arabic} Our submission applies the logistic regressor from 
scikit-learn.%
\footnote{http://scikit-learn.org/stable/}
The utilized features are lexical similarities (Section~\ref{sub:sim}) and  
$n$-Grams (Section~\ref{ssub:ngrams}), together with the predictions obtained 
with our contrastive submission~1 (\cf Section~\ref{sub:contrastive}). 

This submission granted us the first position in the competition, showing a 
particularly high performance when labeling \rel comments.

\paragraph{English A}
The submission applies the linear-kernel SVM for model estimation from 
scikit-learn. We used a one-versus-all approach to account for the fact that 
the learning problem is a multiclass one. We tuned the $C$ hyper-parameter of 
the SVM in order to deal with class imbalance ---by increasing the value of 
$C$, we built more complex classifiers for those classes with less instances. 
% 
The features for this submission consist of lexical and semantic similarities 
(Section~\ref{sub:sim}), context information (Section~\ref{ssub:context}), 
$n$-Grams (Section~\ref{ssub:ngrams}), and heuristics 
(Section~\ref{ssub:heuristics}). In a sort of stacking, the output of our 
rule-based system from the contrastive submission~2 is included as another 
feature. 

This submission obtained the third position in the competition. \pot comments 
showed to be the hardest ones to identify, as the border with 
respect to the rest of the comments is fuzzy. (Indeed, a manual inspection on some 
random comments show that the decision between \good and \pot comments is nearly 
impossible.)

\paragraph{English B}

Following the strategy applied during the manual labeling of the \yes/\no 
questions by the task organizers~\cite{Marquez-EtAl:2015:SemEval}, our approach 
consists of three steps:
\Ni identifying the \good comments among those associated to the question;
\Nii classifying each of them as \yes, \no or \unsure; and 
\Niii aggregating the comment-level classifications into a question-level one. 
The overall answer to a question is that of the majority of the comments. In 
case of draw, we opt for labeling it as \unsure.%
\footnote{\yes, the majority class in the training and development partitions, 
could have been the default answer. Still, we opted for a conservative decision: 
deciding \unsure if no evidence enough was at hand.}
Step \Ni is indeed task A. As for step \Nii, our approach to this task is 
identical as that for English A, but adding the features described in 
Sections~\ref{sub:polarity} and~\ref{sub:profile}.

Differently to the rest of the tasks, our submitted results were obtained with a 
classifier trained on the training data only (the development set was 
neglected). The reason behind this decision was that, when learning was performed on training and development sets,
an unexpected distribution of mostly \yes answers on the test was obtained. 
Such distribution is completely different to that observed in both training and development partitions. 
%The reason behind this decision was obtaining an unexpected 
%distribution of (mostly) \yes answers on the test set (completely different to 
%that observed in both training and development partitions. 
Further experiments carried out after the submission demonstrated that the causes for such an 
unexpected behavior were a buggy implementation of some features and the fact that 
some features were computed on unreliable statistics of the data. 
%based on values for which not enough statistics is available. %to be valuable. % as well as a buggy implementation of some others. 
Further discussion is included in Section~\ref{sec:discussionb}.


\subsection{Contrastive Submissions}
\label{sub:contrastive}

\paragraph{Arabic} 

Our contrastive submission~1 is approached as a ranking problem. Similarity 
$sim(q,c)$, after stopwording and stemming, is computed as 
\begin{equation}
 sim(q,c) = \frac{1}{|q|} \sum_{t\in q\cap c} \omega(t) \enspace ,
 \label{eq:overlap}
\end{equation}
% 
where the empirically-set weight $\omega(t)=1$ if $t$ is a $1$-gram and 
$\omega(t)=4$ if $t$ is a $2$-gram. Given the 5 comments 
$c_1,\ldots,c5\in C$ associated to $q$, the maximum similarity $\max_C sim(q,c)$ 
is mapped to a maximum 100\% similarity and the rest of the scores are mapped 
proportionally. Each comment is assigned a class according to the following 
ranges: [80, 100]\% for \dir, (20,80)\% for \rel, and [0,20]\% for \irel.


As for the contrastive submission~2, we built a binary classifier based on 
logistic regression: \dir or no. The comments are then sorted according to the 
classifier's prediction confidence and the final labels are assigned 
accordingly: \dir for the 1st ranked, \rel for the 2nd ranked, and \irel for the 
rest. Only lexical similarities are included as features (discarding those 
weighted with idf variants).


The performance of these two submissions is comparable to that of the primary 
one, particularly when identifying \rel comments. 

\paragraph{English A}

For our contrastive submission~1, the same machine learning schema as for the 
primary submission is used, but now using 
SVM$^\mathrm{light}$~\cite{Joachims:99}. This toolkit allows us to deal with 
the class imbalance by tuning the $j$ parameter (cost of making mistakes on 
positive examples). This time the $C$ hyper-parameter is set to the default 
value. As we focused on improving the performance on \pot instances, we obtained 
better results on this category, even surpassing the overall performance from 
the primary submission.

Our English contrastive submission~2 operates in the same way as the Arabic 
contrastive submission~1. The applied ranges are the same, but this time they used to 
assign the classes \good, \pot, and \bad. Some heuristics override the so 
generated decisions: $c$ is classified as \good if it includes a URL, starts 
with an imperative verb (\eg \textit{try}, \textit{view}, \textit{contact}, 
\textit{check}), or contains \textit{yes words} (\eg \textit{yes}, \textit{yep}, 
\textit{yup}) or \textit{no words} (\eg \textit{no}, \textit{nooo}, 
\textit{nope}). Comments written by the author of the question or including 
acknowledgments are considered \dial and classified as \bad. 
%   only rule-based features 
% (Section~\ref{ssub:heuristics}) \blue{confirm}



\paragraph{English B}

Our contrastive submission~1 is identical to the primary one, but using both 
training and development data for estimating the model. The reason behind the 
disastrous results is a buggy implementation of some of the polarity features 
(\cf Section~\ref{sub:polarity}) and the lack of statistics for properly 
estimating category-level user profiles (\cf Section~\ref{sub:profile}). 
 
 
The contrastive submission~2 consists of a rule-based system. A comment is 
labeled as \yes if it starts with affirmative words: \textit{yes}, \textit{yep}, 
\textit{yeah}, etc.
% \footnote{\blue{complete it or cite the source for affirmative words; the same 
% for the rest}}
It is labeled as \no if it starts with \textit{no}, \textit{nop}, \textit{nope}, 
etc”, 

