\section{Features Description}
\label{sec:approach}

Most of our approaches are built on top of supervised machine learning, whereas 
a few contrastive submissions were based on rule-based approaches. In this 
section we describe all the different features we considered including 
similarity measures (Section~\ref{sub:sim}), the context in which a comment 
appears 
(Section~\ref{ssub:context}), and the occurrence of certain vocabulary and 
phrase triggers (Sections~\ref{ssub:ngrams} and~\ref{ssub:heuristics}). How and 
where they are applied is discussed in Section~\ref{sec:experiments}.

% \subsection{Arabic task}
% \label{sub:app_arabic}
% 
% \begin{description}
%  \item[Lexical similarity]  Massimo, Hamdy's overlap (same as in A) 
% \end{description}

\subsection{Similarity Measures}
\label{sub:sim}

Our intuition is that the higher the similarity  $sim(q,c)$, the higher the 
likelihood that $c$ is a \good answer. We consider different types.

\subsubsection{Lexical Similarities}

After stopwording, we compute $sim(q, c)$ for word $n$-gram representations of 
$q$ and $c$ ($n=[1,\ldots,4]$), and different $sim$ functions: greedy string 
tiling~\cite{Wise:1996}, longest common subsequences~\cite{Allison:1986}, 
Jaccard coefficient~\cite{Jaccard:1901}, word containment~\cite{Lyon:2001}, and 
cosine similarity (cosine is also computed on lemmas and POS tags, either 
including stopwords or not).

% \blue{Wei}
% We computed three features on the intersection of the tokens in both $q$ and 
% $c$ by considering three term weighting schemes:
Three other similarity measures are computed, weighting the terms by means of 
three 
formul\ae: 
%
\begin{eqnarray}
 sim(q, c)=\sum_{t\in q\cap c} & idf(t) \enspace,		\\
 sim(q, c)=\sum_{t\in q\cap c} & log(idf(t)) \enspace, \enspace \mathrm{and} 
\label{idfvar1}\\
 sim(q, c)=\sum_{t\in q\cap c} & log\left(1 + \frac{|C|}{tf(t)}\right) \enspace 
,
\label{idfvar2}
\end{eqnarray}
% 
where $idf(t)$ is the inverse document frequency~\cite{Jones:1972} of term $t$ 
in the entire Qatar Living dataset, $C$ is the amount of comments in the entire 
collection, and $tf(t)$ is the term frequency of the term in the comment. 
Equations~\ref{idfvar1} and~\ref{idfvar2} are variations of the IDF concept by 
Nallapati~\shortcite{Nallapati:2004}.

Yet another similarity variation is considered (only for task B): the cosine 
similarity between the $tf$-$idf$-weighted vocabulary intersection of $q$ and 
$c$.
% . In this case, the weight associated to each word is its tf-idf.

% where $idf(t)$ represents the inverse document frequency of term $t$ in the 
% entire Qatar Living dataset%
% \footnote{All the \textit{idf} values were computed on the Qatar Living dataset, 
% distributed by the task organizers~\cite{Marquez-EtAl:2015:SemEval}: 
% \url{http://alt.qcri.org/semeval2015/task3/}.},
% $|C|$ represents the amount of comments in the entire collection, and $tf(t)$ 
% represents the term frequency of the term in the comment. These are variations 
% of the \textit{idf} concept by \blue{Salton (1986)} and 
% \blue{Nallapati (2004)}.\footnote{\blue{Wei, please add the proper references}}

\subsubsection{Syntactic Similarity}
\label{sub:syntactic}

Partial tree kernel (PTK) similarity between question and comment syntactic 
trees according to~\cite{Moschitti:2006}. The similarity is computed between 
shallow tree  representations of $q$ and $c$. Such trees have lemmas as leaves, 
each leaf has a parent node representing a part-of-speech tag, and 
part-of-speech nodes are grouped by chunks at the top level.

\subsubsection{Semantic Similarities}
\label{sub:semantic}

We apply three approaches to build  word-embedding vector representations:
\Ni an instance of latent semantic analysis~\cite{croce-previtali:2010:GEMS}, 
trained on the Qatar Living corpus applying a co-occurrence window of size 
$\pm3$ and producing a vector of dimension 250, after SVD reduction (we included 
an instance on the entire vocabulary and nouns only);
\Nii GloVe~\cite{Pennington:2014}, using the pre-trained model \textit{Common 
Crawl (42B tokens)}, with 300 dimensions; and 
\Niii COMPOSES~\cite{Baroni:2014}, using previously-estimated predict vectors of 
400 dimensions.%
\footnote{They are available at \url{http://nlp.stanford.edu/projects/glove/} 
and \url{http://clic.cimec.unitn.it/composes/semantic-vectors.html}}
% ; last visit: Jan 6th, 2015.}
We also experimented with \textit{word2vec}~\cite{Mikolov:2013} vectors 
pre-trained (both with cbow and skipgram) and both word2vec and GloVe with 
vectors trained on Qatar Living data, but we discarded them, as they did not 
contribute positively to our approach. Both $q$ and $c$ are then represented by 
a sum of the vectors corresponding to the words within them (neglecting the 
subject of $c$). To estimate $sim(q,c)$ we compute the cosine similarity. 

\subsection{Context}
\label{ssub:context}

Comments are organized sequentially according to the time line of the comment 
thread. Whether a question includes further comments by $u_q$ (some of them 
acknowledging), more than one comment from the same user, or whether $q$ belongs 
to a category in which a given kind of answer is expected, are important 
factors. Therefore, we consider a set of features that try to describe a 
comment in its context.   

A first subset of context features are boolean indicators exploring the 
following situations:

\begin{itemize}\setlength\itemsep{0em}
\item $c$ is written by $u_q$ (\ie the same user behind $q$),
\item \label{enu:context_ack} 
  $c$ is written by $u_q$ and contains and acknowledgment (e.g.   
  \textit{thank*}, \textit{appreciat*}),
\item \label{enu:context_quest}
  $c$ is written by $u_q$ and includes further questions, and 
\item $c$ is written by $u_q$ and includes no acknowledgments nor further 
questions.
\end{itemize}
% 
A second subset explores whether comment $c$ appears in the proximity of a 
comment by $u_q$. The intuition is that acknowledgment or further questions by 
$u_q$ could be a relevant factor when classifying $c$. Features investigating 
the following occurrences have been developed:

\begin{itemize}\setlength\itemsep{0em}
\item among the comments following $c$ there is one by $u_q$ containing 
an acknowledgment,
\item among the comments following $c$ there is one by $u_q$ not 
containing an acknowledgment,
\item among the comments following $c$ there is one by $u_q$ containing a 
question, and
\item among the comments preceding $c$ there is one by $u_q$ containing a 
question.
\end{itemize}

The value of these features is scaled according to the distance $k$, in number 
of comments, between $c$ and the comment by $u_q$ ($k=\infty$ if no 
comments from $u_q$ exist):
%Our function to represent the relationship between a comment 
%$c_{t-k}$ in time $t-k$ and $c_{q,t}$, given that $t$ is the 
%time of the comment by $u_q$ is as follows:
% 
\begin{equation}
 f(c)=\max \left(0\enspace,\enspace 1.1-(k \cdot 0.1) \right)
\end{equation}
%
%where $k$ is the distance between $c_{t-k}$ and $c_{q,t}$ in the past.

We also tried to model potential dialogues by identifying interlacing comments 
between two users. Our dialogue features rely on identifying comments from
a sequence of users 
\begin{align*}
u_i \rightarrow u_j \rightarrow u_i \rightarrow u_j^*,
\end{align*}
% 
%where $u_i$ and $u_j$ are the authors of $c_i$ and $c_j$. 
Note that comments by other 
users can appear in between this ``pseudo-conversation''. Three features are 
considered, whether a comment is at the beginning, middle, or ending position of 
the pseudo-dialogue. We consider three more features for those cases in which 
$u_q=u_j$. 

We are also interested in realizing whether a user $u_i$ has been particularly 
active in a question. As a result, we consider one boolean feature: whether 
$u_i$ wrote more than one comment in the current stream. Three more features 
identify the first, middle and last comments by $u_i$. One extra feature counts 
the total number of comments written by $u_i$. Moreover we empirically observed 
that the likelihood for a comment to be \good decreases the farther it appears 
from the question. Therefore, we consider one more real-valued feature: 
$\max(20, i)/20$, where $i$ represents the position of the comment in the 
stream.

Finally, Qatar Living includes twenty-six different categories in which a person 
could 
request for information and advice. Some of them tend to include more open 
questions and even invite to discussions on ambiguous topics (e.g., \textit{life 
in Qatar}, \textit{Qatari culture}). Some others require more precise answers 
and allow for less discussion (e.g. \textit{Electronics}, \textit{visas and 
permits}). Therefore, we include one boolean feature per category to consider 
this information. 
 
\subsection{Word $n$-Grams}
\label{ssub:ngrams}

Our intuition is that a properly produced question should allow for the creation 
of \good comments. That is, objective and clear questions would tend to produce 
objective and \good comments. On the other side, subjective or badly formulated 
questions would call for \bad comments or even discussion (\ie dialogues) among 
the users. When talking about comments, they could also include 
specific indicators that trigger a \good or \bad class, regardless of the 
specific question it intends to reply to. 

Our aim is capturing those words or pairs of words which are associated to 
questions and comments in the different classes. Our features are composed of 
$[1,2]$-grams by analyzing independently the question and comments. The weights 
are based on tf-idf on the whole Qatar Living dataset. 

\subsection{Heuristics}
\label{ssub:heuristics}

Exploring the data, we noticed that many \good comments suggested visiting a Web 
site or writing to an email address. Therefore, we included two boolean features 
to verify the presence of URLs or emails in $c$. Another feature captures the 
length of $c$, as longer (\good) comments usually contain detailed information 
to answer a question. 



% \blue{TODO complete these}
% 
% \begin{itemize}
%  \item A boolean feature, whether $c$ contains a URL or electronic mail. 
%  \item the length of $c_i$ in characters, as we empirically observed that long 
%   comments tend to be \good.
% \blue{simone}
% \end{itemize}
% 
% 
% \blue{Hamdy's contrastive}
% Our contrastive submission \blue{x} is a rule-based system. A comment is 
% labeled as \good if starts with one of a set of imperative verbs, including 
% \textit{try}, \textit{view}, \textit{contact}, \textit{check}%
% \footnote{
% % yes list: {"yes", "yep", "yup", "yap", "yeah", "yea", "ya", "yess", 
% "yeh", "sure"} --> both yes and good
% 2- no list: {"no", "noo", "nooo", "nop", "nope"}
%  --> both no and good
% % 3- thanks/dialogue list: {"thank", "thx", "thanks", "thanx", "thnk", "tnx", 
% "thnak", "sorri", "welcom", "wow"} --> dialogues
% 4. generic answers list: {"check", "try", "go", "call", "contact", "follow", 
% "go", "take", "talk", "use", "visit", "watch"} --> good


%HAMDYS
%ENGLISH:
% 11. if score == highest score                                          -> Good
% 12. if score >= 0.5 of the second highest score             -> Good
% 13. if score == 0                                                              
% -> Bad  Otherwise                                                         
%           -> Potential
% [12:35:11 PM] Hamdy Mubarak: Porter Stemmer problems:
% - it gives incorrect stem when word starts with capital letter (at the beginning 
% of sentence). ex: Lady, Ladies and lady will give different stems
% - stem is not always correct when a named entity written in small letters, ex: 
% Los angeles.
% - it doesn't have a built-in spell checker to handle spelling mistakes
% 
% I used a list of ~30,000 words and their stems (lookup table):
% http://snowball.tartarus.org/algorithms/english/diffs.txt
% 
% The training data is stemmed using the word list, and stopwords are marked by 
% revising the top 3,000 words (freq >= 15)
% 
% 
% \blue{complete it or cite the source for affirmative words; the same 
% for the rest}}, ..” and includes a URL or phone number. A comment is labeled 
% as \dial if it starts with \textit{thanks}, \textit{thx}, \textit{thanx},..” 
% or it has been written by the same person that asked the question.%
% \footnote{\blue{I am tempted to include a simple table with all the 
% vocabularies in these rules. TODO check these vocabularies}}
% 


% \bsegin{description}
%  \item[Lexical similarity] 
%  \item[Syntactic similarity] Massimo's PTK
%  \item[Semantic similarity] (Preslav, Simone's LSA)
%  \item[Context-based] (Simone)
%  \item[Heuristics] Hamdy's
% \end{description}

\subsection{Polarity}
\label{sub:polarity}

These features, used in task B only, intend to determine whether a comment is 
positive or negative, which could be associated to \yes or \no answers. A 
quantitative polarity of $c$ is modeled as:
\begin{equation}
pol(c) = \sum_{w\in c} pol(w) 
\end{equation}
%
where $pol(w)$ is the polarity of word $w$ in the NRC Hashtag Sentiment 
Lexicon v0.1~\cite{MohammadKZ2013}.%
\footnote{\url{%
http://www.umiacs.umd.edu/~saif/WebPages/Abstracts/NRC-SentimentAnalysis.htm}; 
last visit: Jan 18, 2015.}
Words with polarity in the range $(-1,1)$ are discarded to neglect nearly 
neutral words.

We consider other boolean features on the existence of some keywords in the 
comment. Features are set to true if $c$ contains
\Ni \textit{yes}, \textit{can}, \textit{sure}, \textit{wish}, \textit{would} or
\Nii \textit{no}, \textit{not}, \textit{neither}.
% ; or 
% \Niii  a \textit{URL}.

% \blue{together with others intending to model \yes, \no, and \unsure answers.}

\subsection{User's Profile}
\label{sub:profile}

With this set of features we aim at modeling the behavior of the different 
participants in previous queries. Given comment $c$ by user $u$, we consider 
the number of \good, \bad, \pot, and \dial comments the user has produced 
before. We also consider the average word length of \good, \bad, \pot, and \dial 
comments. These features are computed both considering all the questions and 
only those from the same category as the current one.%
\footnote{In Section~\ref{sec:discussionb} we will observe that computing these 
category-level statistics was not a good idea.}


% \blue{these two features were under consideration already:}
% length of the comment, and the inverse rank of the comment in the list of all 
% comments for a question.


% \begin{description}
%  \item[Lexical similarity]  Iman
%  \item[Heuristics] Iman's ``lexical features''
%  \item[Sentiment] Iman
%  \item[Context] Iman's user profiles
% \end{description}

