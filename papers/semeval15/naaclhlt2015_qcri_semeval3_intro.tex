\section{Introduction}
\label{sec:intro}


The SemEval-2015 Task 3 ---Answer Selection in Community Question Answering---, 
challenged participants in the problem of automatically identifying the 
appropriateness of user-generated answers in a community question answering 
setting both in Arabic and English~\cite{Marquez-EtAl:2015:SemEval}. A question 
$q\in Q$, asked by user $u_q$, together with a set of comments $C$ are given and 
the system is asked to determine whether a comment $c\in C$ offers a suitable 
answer to $q$ or not. 

In the case of Arabic, the questions were extracted from \textit{Fatwa}, a 
community question answering website about Islam.%
\footnote{\url{http://fatwa.islamweb.net}} 
Each question includes five comments, provided by scholars on 
the topic, each of which has to be automatically labeled as 
\Ni \dir: a direct answer to the question;
\Nii \rel: not a direct answer to the question but with information related to 
the topic; and 
\Niii \irel: an answer to another question, not related to the topic. 

In the case of English, the dataset was extracted from \textit{Qatar Living}, 
a forum for people to pose questions on multiple aspects of daily life in 
Qatar.%
\footnote{\url{http://www.qatarliving.com/forum}}
Unlike \textit{Fatwa}, the questions and comments in this dataset come from 
regular users, making them significantly more varied, informal, open, and noisy. 
In this case, the input to the system consists of a question and a variable 
number of comments, each of which are to be labeled as 
\Ni \good: the comment is definitively relevant; 
\Nii \pot: the comment is potentially useful; and 
\Niii \bad: the comment is irrelevant (\eg it is part of a dialogue, unrelated 
to the topic, or it is written in a language other than English). 
We refer to this task as English task A. Additionally, a subset of the questions 
in the corpus requires a \yes/\no answer.
% , which means the expected answer to the question is precisely either \yes or 
% \no. 
In this case, the task consists of determining whether the overall answer to 
the question, according to the evidence provided by the comments, is 
\Ni \yes, 
\Nii \no, or 
\Niii \unsure when there is no evidence to make a decision. 
We refer to this as English task B. 
Refer to~\cite{Marquez-EtAl:2015:SemEval} for more information on tasks 
definitions and settings.

In this paper, we describe the supervised machine learning approach of QCRI\@. 
We approach the problem as a classification task considering different kinds 
of features: lexical, syntactic and semantic similarities, the context in which 
a comment appears, $n$-grams occurrence, and some heuristics on specific 
keywords. Our approach ranked 1st out of four teams in the Arabic task, 3rd out 
of twelve in English task~A, and 3rd out of eight in English task~B. 

The rest of the paper is organized as follows. Section~\ref{sec:approach} 
describes the features used in our approaches. Section~\ref{sec:experiments} 
describes our prediction models and discusses the results obtained at 
competition time. Section~\ref{sec:discussion} discusses further 
post-competition experiments and offers some final remarks.