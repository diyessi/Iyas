%
% File naaclhlt2015.tex
%

\documentclass[11pt,letterpaper]{article}
\usepackage{naaclhlt2015}
\usepackage{times}
\usepackage{latexsym}
\setlength\titlebox{6.5cm}    % Expanding the titlebox

%USER-DEFINED PACKAGES
\usepackage{url}
\usepackage{color}
\usepackage{amsmath} % for the "align" and "align*" environments

%\usepackage{chato-notes}


%colors for highlighting
\newcommand{\blue}{\textcolor{blue}}
\newcommand{\red}{\textcolor{red}} 
\newcommand{\green}{\textcolor{green}} 

\newcommand{\cf}{cf.~}
\newcommand{\ie}{i.e.,~}
\newcommand{\eg}{e.g.,~}

\newcommand{\Ni}{({\em i})~}
\newcommand{\Nii}{({\em ii})~}
\newcommand{\Niii}{({\em iii})~}
\newcommand{\Niv}{({\em iv})~}

\newcommand{\good}{\texttt{GOOD}\,}
\newcommand{\bad}{\texttt{BAD}\,}
\newcommand{\dial}{\texttt{DIALOGUE}\,}
\newcommand{\pot}{\texttt{POTENTIAL}\,}

\newcommand{\dir}{\texttt{DIRECT}\,}
\newcommand{\rel}{\texttt{RELATED}\,}
\newcommand{\irel}{\texttt{IRRELEVANT}\,}

\newcommand{\yes}{\texttt{YES}\,}
\newcommand{\no}{\texttt{NO}\,}
\newcommand{\unsure}{\texttt{UNSURE}\,}


\title{%
% QCRI: Experiments in Answer Selection for \\Community Question Answering 
% for Arabic and English
QCRI: Answer Selection for Community Question Answering --\\
Experiments for Arabic and English
% \Thanks{}
}

\author{%
Massimo Nicosia$^1$, Simone Filice$^2$, Alberto Barr\'on-Cede\~no$^2$,  \\
{\bf Iman Saleh}$^3$, {\bf Hamdy Mubarak}$^2$, {\bf Wei Gao$^2$}, 
{\bf Preslav Nakov$^2$}, \\
{\bf Giovanni Da San Martino$^2$}, {\bf Alessandro Moschitti}$^2$,
{\bf Kareem Darwish$^2$}, \\ 
{\bf Llu\'is M\`arquez$^2$}, {\bf Shafiq Joty$^2$} \and {\bf Walid Magdy$^2$} 
\\
$^1$ University of Trento \hspace{1em}
$^2$ Qatar Computing Research Institute	\hspace{1em}
$^3$ Cairo University \\
\small
{\tt massimo.nicosia@unitn.it}	\\
\small
{\tt \{sfilice,albarron,hmubarak,wgao,pnakov,gmartino\}@qf.org.qa} 	\\
\small
{\tt \{amoschitti,kdarwish,lmarquez,sjoty,wmagdy\}@qf.org.qa}	\\
\small
{\tt iman.saleh@fci-cu.edu.eg}
% \And 
%  \\
% Qatar Computing Research Institute \\
% {\tt }
% 
% Massimo Nicosia \\ 
% University of Trento \\
% {\tt massimo.nicosia@unitn.it}
% \And 
% Simone Filice \and Alberto Barr\'on-Cede\~no \\
% Qatar Computing Research Institute \\
% {\tt \{sfilice , albarron\}@qf.org.qa}
% \AND
% Iman Saleh \\
% Cairo University \\
% {\tt iman.saleh@fci-cu.edu.eg}
% \And
% Hamdy Mubarak	\\
% Qatar Computing Research Institute	\\
% {\tt hmubarak@qf.org.qa}
% 
% 
% 
% 
}


% \author{Author 1\\
% 	    XYZ Company\\
% 	    111 Anywhere Street\\
% 	    Mytown, NY 10000, USA\\
% 	    {\tt author1@xyz.org}
% 	  \And
% 	Author 2\\
%   	ABC University\\
%   	900 Main Street\\
%   	Ourcity, PQ, Canada A1A 1T2\\
%   {\tt author2@abc.ca}}

\date{}

\begin{document}
\maketitle
\begin{abstract}
This paper describes QCRI's participation in SemEval-2015 Task 3 ``Answer 
Selection in Community Question Answering'',
which targeted real-life Web forums,
and was offered in both Arabic and English.
We apply a supervised machine learning approach 
considering a manifold of features including among others word $n$-grams, text 
similarity, sentiment analysis, the presence of specific words, and the context 
of a comment.
Our approach was the best performing one in the Arabic subtask and the third 
best in the two English subtasks.
\end{abstract}

\section{Introduction}
\label{sec:intro}


The SemEval-2015 Task 3 ---Answer Selection in Community Question Answering---, 
challenged participants in the problem of automatically identifying the 
appropriateness of user-generated answers in a community question answering 
setting both in Arabic and English~\cite{Marquez-EtAl:2015:SemEval}. A question 
$q\in Q$, asked by user $u_q$, together with a set of comments $C$ are given and 
the system is asked to determine whether a comment $c\in C$ offers a suitable 
answer to $q$ or not. 

In the case of Arabic, the questions were extracted from \textit{Fatwa}, a 
community question answering website about Islam.%
\footnote{\url{http://fatwa.islamweb.net}} 
Each question includes five comments, provided by scholars on 
the topic, each of which has to be automatically labeled as 
\Ni \dir: a direct answer to the question;
\Nii \rel: not a direct answer to the question but with information related to 
the topic; and 
\Niii \irel: an answer to another question, not related to the topic. 

In the case of English, the dataset was extracted from \textit{Qatar Living}, 
a forum for people to pose questions on multiple aspects of daily life in 
Qatar.%
\footnote{\url{http://www.qatarliving.com/forum}}
Unlike \textit{Fatwa}, the questions and comments in this dataset come from 
regular users, making them significantly more varied, informal, open, and noisy. 
In this case, the input to the system consists of a question and a variable 
number of comments, each of which are to be labeled as 
\Ni \good: the comment is definitively relevant; 
\Nii \pot: the comment is potentially useful; and 
\Niii \bad: the comment is irrelevant (\eg it is part of a dialogue, unrelated 
to the topic, or it is written in a language other than English). 
We refer to this task as English task A. Additionally, a subset of the questions 
in the corpus requires a \yes/\no answer.
% , which means the expected answer to the question is precisely either \yes or 
% \no. 
In this case, the task consists of determining whether the overall answer to 
the question, according to the evidence provided by the comments, is 
\Ni \yes, 
\Nii \no, or 
\Niii \unsure when there is no evidence to make a decision. 
We refer to this as English task B. 
Refer to~\cite{Marquez-EtAl:2015:SemEval} for more information on tasks 
definitions and settings.

In this paper, we describe the supervised machine learning approach of QCRI\@. 
We approach the problem as a classification task considering different kinds 
of features: lexical, syntactic and semantic similarities, the context in which 
a comment appears, $n$-grams occurrence, and some heuristics on specific 
keywords. Our approach ranked 1st out of four teams in the Arabic task, 3rd out 
of twelve in English task~A, and 3rd out of eight in English task~B. 

The rest of the paper is organized as follows. Section~\ref{sec:approach} 
describes the features used in our approaches. Section~\ref{sec:experiments} 
describes our prediction models and discusses the results obtained at 
competition time. Section~\ref{sec:discussion} discusses further 
post-competition experiments and offers some final remarks.
% % \section{Tasks Definition}
% \label{sec:task}
\section{Approach}
\label{sec:approach}

\blue{TODO describe the datasets?}
Our approach performs multiclass classification on the basis of a one-vs-rest 
support vector machines strategy (\ie we train one classifier for each class). 
Our learning process consists of 10-fold cross-validation on the training set 
tuning $F$-measure. 
\blue{\ldots} 
Each comment $c$ attached to question $q$ is represented by a features' vector 
including similarities (Section~\ref{ssub:sim}), the context in which a comment 
appears (Section~\ref{ssub:context}), and the occurrence of certain vocabulary 
and phrase triggers (Sections~\ref{ssub:ngrams} and~\ref{ssub:heuristics}). 
% Our classifications are made at comment level of comment (except for task B, 
% \cf Section~\ref{sub:app_enB}). 


\subsection{Arabic task}
\label{sub:app_arabic}

\begin{description}
 \item[Lexical similarity]  Massimo, Hamdy's overlap (same as in A) 
\end{description}

\blue{We still have to identify what features are computed for each language 
(\eg semantic are not considered in Arabic, I guess)}

\subsubsection{Similarities}
\label{ssub:sim}

Our intuition is that the higher the similarity between $c$ and $q$ is, the 
more likely is that $c$ represents a \good answer. Following, we describe the 
different types of similarities $sim(q,c)$ we compute.

\paragraph{Lexical similarities}
\blue{Massimo}
We compute $sim(q, c)$ for word $n$-gram representations of the question and 
comment ($n=[1,\ldots,4]$) and different $sim$ functions including: greedy 
string tiling~\blue{ref}, longest common subsequences~\blue{ref}, Jaccard 
coefficient~\cite{Jaccard:1901}, containment~\blue{ref}, and cosine similarity 
(cosine is also computed on lemmas and POS tags, either including stopwords or 
not).%
\footnote{\blue{So the rest are computed on the original tokens? Please, 
clarify}}


% \blue{Wei}
% We computed three features on the intersection of the tokens in both $q$ and 
% $c$ by considering three term weighting schemes:
Three other (pseudo-)similarities are computed, by weighting the terms with idf 
variations, by means of three formul\ae: 
%
\begin{eqnarray}
 sim(q, c)=\sum_{t\in q\cap c} & idf(t) \enspace,		\\
 sim(q, c)=\sum_{t\in q\cap c} & log(idf(t)) \enspace, \enspace \mathrm{and} 
\label{idfvar1}\\
 sim(q, c)=\sum_{t\in q\cap c} & log\left(1 + \frac{|C|}{tf(t)}\right) 
\label{idfvar2}
\end{eqnarray}
% 
where $idf(t)$ represents the inverse document frequency~\cite{Jones:1972} 
of term $t$ in the entire Qatar Living dataset\footnote{keeps your content}, 
$C$ represents the amount of comments in the entire collection, and $tf(t)$ 
represents the term frequency of the term in the comment. 
Equations~\ref{idfvar1} and~\ref{idfvar2} are variations of the IDF 
concept by Nallapati~\shortcite{Nallapati:2004}.







% where $idf(t)$ represents the inverse document frequency of term $t$ in the 
% entire Qatar Living dataset%
% \footnote{All the \textit{idf} values were computed on the Qatar Living dataset, 
% distributed by the task organizers~\cite{Marquez-EtAl:2015:SemEval}: 
% \url{http://alt.qcri.org/semeval2015/task3/}.},
% $|C|$ represents the amount of comments in the entire collection, and $tf(t)$ 
% represents the term frequency of the term in the comment. These are variations 
% of the \textit{idf} concept by \blue{Salton (1986)} and 
% \blue{Nallapati (2004)}.\footnote{\blue{Wei, please add the proper references}}







\paragraph{Syntactic similarity}
\label{ssub:syntactic}

\blue{Massimo}
Partial tree kernel (PTK) similarity between question and comment according 
to~\cite{Moschitti:2006}. \blue{add some details}

\paragraph{Semantic similarities}
\label{ssub:semantic}

We use word-embedding vector representations, including three approaches:
\Ni an instance of latent semantic analysis~\cite{croce-previtali:2010:GEMS}, 
trained on the Qatar Living corpus applying a co-occurrence window of size 
$\pm3$ and coming out with a vector of dimension 250, after SVD reduction (we 
included an instance on the entire vocabulary and nouns only);
\Nii GloVe~\cite{Pennington:2014}, using the pre-trained model \textit{Common 
Crawl (42B tokens)}, with 300 dimensions;%
\footnote{Available at \url{http://nlp.stanford.edu/projects/glove/}; last 
visit: Jan 6th, 2015.}
and \Niii COMPOSES~\cite{Baroni:2014}, using previously-estimated predict 
vectors of 400 dimensions.%
\footnote{Available at 
\url{http://clic.cimec.unitn.it/composes/semantic-vectors.html}; last visit: Jan 
6th, 2015.}
We also experimented with \textit{word2vec}~\cite{Mikolov:2013} 
vectors pre-trained (both with cbow and skipgram) and both word2vec and GloVe 
with vectors trained on Qatar Living data, but we discarded them, as they did 
not contribute positively to our approach.
Both $q$ and $c$ are then represented by a sum of the vectors 
corresponding to the words within them (neglecting the subject of $c$), and 
compute the cosine similarity to estimate $sim(q,c)$. 
\footnote{\blue{Preslav, we have some extra details for LSA, such as the 
window size, etc. If you give more details for your embeddings, we can 
improve both descriptions}}

\blue{These semantic similarities are not applied in the Arabic task.}

\subsubsection{Context \blue{Simone/Alberto}}
\label{ssub:context}

Intuitively, whether a question includes further comments by $u_q$ (some of 
them acknowledging), more than one comment from the same user, or whether $q$ 
belongs to a category in which a given kind of answer is expected, are important 
factors when classifying a comment. Therefore, we consider set of features that 
try to describe a comment in its context.   

Let $C={c_1, \ldots,c_C}$ be the stream of comments associated to question $q$, 
asked by $u_q$. The features for comment $c$ in the first subset are of type 
boolean. The first four of them are set to \texttt{True} according to the 
following criteria:

\begin{enumerate}
\item $c$ is written by $u_q$ (\ie the same user behind $q$); 
\item \label{enu:context_ack} 
  $c$ is written by $u_q$ and contains and acknowledgment (e.g.   
  \textit{thank*}, \textit{appreciat*});
\item \label{enu:context_quest}
  $c$ is written by $u_q$ and includes further questions; and
\item $c$ is written by $u_q$ and includes no acknowledgments nor further 
questions.
\end{enumerate}
% 
Our second subset of context-based features intends to model a comment according 
to those comments by $u_q$ appearing in its proximity. Intuitively, whether $c$ 
appears close to an acknowledgment or further questions by $u_q$ could be a 
relevant factor when classifying it. Our function to represent the relation 
between a comment $c_{t-k}$ in time $t-k$ and $c_{q,t}$, given that $t$ 
is the time of the comment by $u_q$ is as follows:
% 
\begin{equation}
 f(c_{t-k})=\max \left(1.1-(k*0.1) , 0 \right)
\end{equation}
%
where $k$ is the distance between $c_{t-k}$ and $c_{q,t}$ in the past and a 
stop criterion exists: the occurrence of another comment by $u_q$. This function 
is applied to generate four features according to four criteria:

\begin{enumerate}\setcounter{enumi}{4}
\item a $c_q$ for which feature~\ref{enu:context_ack} is \texttt{True},
\item a $c_q$ for which feature~\ref{enu:context_ack} is \texttt{False}, 
\item a $c_q$ for which feature~\ref{enu:context_quest} is \texttt{True}, and 
\item same as the previous one, but looking at the future instead. 
\end{enumerate}


We also tried to model potential dialogues by identifying interlacing comments 
between two users. Our dialogue features rely on identifying 
a sequence of comments 
\begin{align*}
c_i \rightarrow c_j \rightarrow c_i \rightarrow c_j^*,
\end{align*}
% 
where $u_i$ and $u_j$ are the authors of $c_i$ and $c_j$. 
Note that comments by other 
users can appear in between this ``pseudo-conversation''. Three features are 
considered, whether a comment is at the beginning, middle, or ending position of 
the pseudo-dialogue. We consider three more features for those cases in which 
$q=j$. 

We are also interested in realizing whether a user $u_i$ has been particularly 
active in a question. As a result, we consider one boolean feature, whether 
$u_i$ wrote more than one comment in the current stream, and three more features 
identifying the first, middle and last comments by $u_i$. One extra real feature 
counts the total number of comments written by $u_i$.

Qatar Living includes twenty-six different categories in which a person could 
request for information and advice. Some of them tend to include more open 
questions and even invite to discussions on ambiguous topics (e.g., \textit{life 
in Qatar}, \textit{Qatari culture}). Some others require more precise answers 
and allow for less discussion (e.g. \textit{Electronics}, \textit{visas and 
permits}). Therefore, we include one boolean feature per category to consider 
this information. 
 
We empirically observed that the likelihood for a comment to be \good decreases 
the farther it appears from the question. Therefore, we consider one more 
real-valued feature: $\max(20, i)/20$, where $i$ represents the position of 
the comment in the stream.
 
% \Ni . further developed some “context” features that analyze the whole 
% stream of 
% comments within a question to investigate manifold aspects such as the presence 
% of an acknowledgment or a new question from the question author (evidence of at 
% least a good comment among the previous ones), the presence of multiple 
% comments 
% from the same user or the interlacement of comments between two or more users 
% (evidence of a dialogue).



\subsubsection{$n$-Grams \blue{Massimo}}
\label{ssub:ngrams}

Our intuition is that a properly produced question should allow for the 
creation of \good comments. That is, objective and clear questions would tend to 
produce objective and \good comments. On the other side, subjective or badly 
formulated questions would call for \bad comments or even discussion (\ie 
dialogues) among the users. When talking about comment, they could also include 
specific indicators that trigger a \good or \bad class, regardless of the 
specific question it intends to reply to. The aim is capturing those 
$[1,2]$-grams which are associated to questions and comments in the different 
classes.

Our features are composed of $[1,2]$-grams by analyzing independently the 
question and comments. The weights are based on tf-idf on the whole Qatar 
Living 
dataset. 

\subsubsection{Heuristics}
\label{ssub:heuristics}

\begin{itemize}
 \item A boolean feature, whether $c$ contains a URL or electronic mail. 
 \item the length of $c_i$ in characters, as we empirically observed that long 
  comments tend to be \good.
\blue{simone}
\end{itemize}


\blue{Hamdy's contrastive}
Our contrastive submission \blue{x} is a rule-based system. A comment is 
labeled as \good if starts with one of a set of imperative verbs, including 
\textit{try}, \textit{view}, \textit{contact}, \textit{check}%
\footnote{
% yes list: {"yes", "yep", "yup", "yap", "yeah", "yea", "ya", "yess", 
% "yeh", "sure"} --> both yes and good
% 2- no list: {"no", "noo", "nooo", "nop", "nope"}
 --> both no and good
% 3- thanks/dialogue list: {"thank", "thx", "thanks", "thanx", "thnk", "tnx", 
% "thnak", "sorri", "welcom", "wow"} --> dialogues
% 4. generic answers list: {"check", "try", "go", "call", "contact", "follow", 
% "go", "take", "talk", "use", "visit", "watch"} --> good


%HAMDYS
%ENGLISH:
% 11. if score == highest score                                          -> Good
% 12. if score >= 0.5 of the second highest score             -> Good
% 13. if score == 0                                                              
% -> Bad  Otherwise                                                         
%           -> Potential
% [12:35:11 PM] Hamdy Mubarak: Porter Stemmer problems:
% - it gives incorrect stem when word starts with capital letter (at the beginning 
% of sentence). ex: Lady, Ladies and lady will give different stems
% - stem is not always correct when a named entity written in small letters, ex: 
% Los angeles.
% - it doesn't have a built-in spell checker to handle spelling mistakes
% 
% I used a list of ~30,000 words and their stems (lookup table):
% http://snowball.tartarus.org/algorithms/english/diffs.txt
% 
% The training data is stemmed using the word list, and stopwords are marked by 
% revising the top 3,000 words (freq >= 15)


\blue{complete it or cite the source for affirmative words; the same 
for the rest}}, ..” and includes a URL or phone number. A comment is labeled 
as \dial if it starts with \textit{thanks}, \textit{thx}, \textit{thanx},..” 
or it has been written by the same person that asked the question.%
\footnote{\blue{I am tempted to include a simple table with all the 
vocabularies in these rules. TODO check these vocabularies}}



% \bsegin{description}
%  \item[Lexical similarity] 
%  \item[Syntactic similarity] Massimo's PTK
%  \item[Semantic similarity] (Preslav, Simone's LSA)
%  \item[Context-based] (Simone)
%  \item[Heuristics] Hamdy's
% \end{description}

\subsection{English Task B}
\label{sub:app_enB}

Following the strategy applied during the manual labelling by the task 
organizers~\blue{ref}, our approach to task B is divided in three steps:
\Ni identifying the \good comments among those associated to the question;
\Nii classifying each of the \good comments as \yes, \no, or \unsure; and 
\Niii voting%
\footnote{\blue{I'm sure Simone has the ``posh'' word for this}} to determine 
the overall class of the question. The overall answer to a question is that of 
the majority of the comments. In case of draw, we opt for labeling it as 
\unsure.%
\footnote{Alternatively, \yes could have been the default answer, as this is 
clearly the majority class in the training and development partitions. Still, 
we opted for a conservative decision: opting for \unsure if no evidence enough 
is at hand.}

Step \Ni is indeed task A. As for step \Nii, we consider again all the 
features used  for task A, toghether with others intending to model \yes, 
\no, and \unsure answers. Our intention is determining whether a comment is 
considered positive or negative, the existence of key elements for supporting 
and answer, such as a URL, and even the profile of the user of every comment, 

We also compute the sentiment score of a comment by analyzing its polarity. We 
model this function as:
\begin{equation}
pol(c) = \sum_{w\in c} pol(w) 
\end{equation}
%
where $pol(w)$ represents the polarity of word $w$ in the lexicon 
\blue{XYZ (ref)}. In order to neglect nearly neutral words, we discard those 
wich polarity is in the range $(-1,1)$. 


Additionaly to a content-based polarity, we also exploit what we call a user 
profile. Given comment $c$ by user $u$, we consider the number of 
\good, \bad, \pot, and \dial comments the user has produced before. We also 
consider the average word length of \good, \bad, \pot, and \dial comments.
These features are computed considering only those previous questions from 
the same category as the current one. 

We also compute a variation of the cosine similarity in which only the 
vocabulary intersection between $q$ and $c$ is considered \blue{why?}. The 
weight associated to each word is the tfidf, for which the IDF values were 
computed considering the entire Qatar Living dataset.


Heuristics are also applied on the existence of some keywords in the comment. 
Features are set to true if $c$ contained
\Ni \textit{yes}, \textit{can}, \textit{sure}, \textit{wish}, \textit{would};
\Nii \textit{no}, \textit{not}, \textit{neither}; or 
\Niii  a \textit{URL}.





\blue{these two features were under consideration already:}
length of the comment, and the inverse rank of the comment in the list of all 
comments for a question.


% \begin{description}
%  \item[Lexical similarity]  Iman
%  \item[Heuristics] Iman's ``lexical features''
%  \item[Sentiment] Iman
%  \item[Context] Iman's user profiles
% \end{description}


\subsubsection{Hamdy's contrastive}

Our contrastive submission \blue{X} is a rule-based system. A comment is 
labeled as \yes if it starts with affirmative words: \textit{yes}, 
\textit{yep}, \textit{yeah}, etc.%
\footnote{\blue{complete it or cite the source for affirmative words; the same 
for the rest}}
It is labeled as \no if it starts with \textit{no}, \textit{nop}, 
\textit{nope}, etc”, 


\section{Submissions and Results}
\label{sec:experiments}

Now we describe our primary submissions to the three tasks, followed by the 
contrastive submissions. Table~\ref{tab:results} includes our official 
competition results; all the reported $F_1$ values are macro-averaged.

\subsection{Primary Submissions}

In general, our approaches perform multi-class classification on the basis of a 
one-vs-rest support vector machines strategy (\ie we train one classifier for 
each class). Our classifications for both Arabic and English A are at the
comment level.

\begin{table}%[h]
\centering
\footnotesize
%  \begin{tabular}{|l|c@{\hskip 0.2cm}c@{\hskip 0.2cm}c@{\hskip 0.2cm}c|}
\begin{tabular}{|l|cccc|}
  \hline
  \bf ar	& \dir & \texttt{IRREL} & \rel & \texttt{F$_1$}\\  \hline  
  primary	& $77.31$ & $91.21$	& $67.13$	&  $78.55$ \\
  cont$_1$	& $74.89$ & $91.23$	& $63.68$	&  $76.60$ \\
  cont$_2$	& $76.63$ & $90.30$	& $63.98$	& $76.97$ \\  
%   primary	& $77.31$ & $91.21$	& $67.13$	& $78.55$ \\
%   cont$_1$	& $74.89$ & $91.23$	& $63.68$	& $76.60$ \\
%   cont$_2$	& $76.63$ & $90.30$	& $63.98$	& $76.97$ \\
  \hline \hline

  \bf en A	& \good   & \bad 	& \texttt{POT}	& \texttt{F$_1$}\\\hline
  primary	& $78.45$ & $72.39$	& $10.40$	& $53.74$ \\
  cont$_1$ 	& $76.08$ & $75.68$	& $17.44$	& $56.40$ \\
  cont$_2$ 	& $75.46$ & $72.48$ 	& $\,\,\,7.97$	& $51.97$ \\
\hline  \hline

\bf en B	& \yes	  & \no		& \unsure	& F$_1$	 \\
  \hline  
  primary	& $80.00$ & $44.44$	& $36.36$	& $53.60$ \\
  cont$_1$ 	& $75.68$ & $\,\,\,0.00$& $\,\,\,0.00$	& $25.23$ \\
  cont$_2$ 	& $66.67$ & $33.33$ 	& $47.06$	& $49.02$ \\
  \hline
 \end{tabular}
\caption{Per-class and macro-averaged $F_1$-measure of our primary and 
contrastive submissions to SemEval Task 3 for Arabic (ar) and English 
(en) A and B.
\label{tab:results}}
\end{table}

\paragraph{Arabic} 
Our submission applies the logistic regressor from scikit-learn.%
\footnote{http://scikit-learn.org/stable/}
The features are lexical similarities (Section~\ref{sub:sim}) and  $n$-grams 
(Section~\ref{ssub:ngrams}). In a sort of stacking, the output of our 
contrastive submission~1 is included as another feature (\cf 
Section~\ref{sub:contrastive}). 

This submission achieved the first position in the competition (F$_1=78.55$, 
compared to $70.99$ for the second one). It showed a particularly high 
performance when labeling \rel comments.

\paragraph{English A}
The submission applies a linear-kernel SVM from scikit-learn. We used a 
one-versus-rest approach to account for the fact that the learning problem is a 
multiclass one. We tuned the value of the $C$ hyper-parameter of the SVM in 
order to deal with class imbalance ---by increasing the value of $C$, we built 
more complex classifiers for those classes with less instances. The features for 
this submission consist of lexical, syntactic, and semantic similarities 
(Section~\ref{sub:sim}), context information (Section~\ref{ssub:context}), 
$n$-grams (Section~\ref{ssub:ngrams}), and heuristics 
(Section~\ref{ssub:heuristics}). Similarly to the Arabic submission, the output 
of our rule-based system from the contrastive submission~2 is another feature. 

This submission achieved the third position in the competition (F$_1=53.74$, 
compared to $57.19$ for the top one). \pot comments showed to be the 
hardest ones to identify, as the border with respect to the rest of the comments 
is very fuzzy. Indeed, a manual inspection on some random comments show that the 
decision between \good and \pot comments is nearly impossible in some cases.

\paragraph{English B}

Following the manual labeling strategy applied to the \yes/\no questions by the 
task organizers~\cite{Marquez-EtAl:2015:SemEval}, our approach consists of three 
steps:
\Ni identifying the \good comments among those associated with $q$;
\Nii classifying each of them as \yes, \no, or \unsure; and 
\Niii aggregating the comment-level classifications into question-level. The 
overall answer to $q$ becomes that of the majority of the comments. In case of a
draw, we opt for labeling it as \unsure.%
\footnote{The majority class in the training and dev.\ sets (\yes), could have 
been the default answer. Still, we opted for a conservative decision: deciding 
\unsure if no evidence enough was at hand.}
Step \Ni is indeed task A. As for step \Nii, we train a classifier as that for 
English task A, but adding the polarity and user's profile features (\cf 
Sections~\ref{sub:polarity} and~\ref{sub:profile}).%
\footnote{Even if the user's profile information seems to fit with task A, 
rather than B, at development time they showed to be effective only for the 
latter one.}

This submission achieved the third position in the competition (F$_1=53.60$, 
compared to $63.70$ for the top one). Differently to the rest of the tasks, our 
submitted results were obtained with a classifier trained on the training data 
only (the development set was neglected). The reason behind this decision was 
that we obtained an unexpected distribution of mostly \yes predictions on the 
test set when both training and development sets had been considered. Such 
distribution is completely different to that observed in both training and 
development partitions. Further experiments, carried out after the submission, 
demonstrated that the causes for such an unexpected behavior were bugs in the 
implementation of some features and the fact that some features were computed on 
unreliable statistics of the data. Further discussion is included in 
Section~\ref{sec:discussionb}.


\subsection{Contrastive Submissions}
\label{sub:contrastive}

\paragraph{Arabic} 

We approach our contrastive submission~1 as a ranking problem. After 
stopwording and stemming, $sim(q,c)$ is computed as 
\begin{equation}
 sim(q,c) = \frac{1}{|q|} \sum_{t\in q\cap c} \omega(t) \enspace ,
 \label{eq:overlap}
\end{equation}
% 
where the empirically-set weight $\omega(t)=1$ if $t$ is a $1$-gram and 
$\omega(t)=4$ if $t$ is a $2$-gram. Given the 5 comments $c_1,\ldots,c_5\in C$ 
associated to $q$, the maximum similarity $\max_C sim(q,c)$ is mapped to a 
maximum 100\% similarity and the rest of the scores are mapped proportionally. 
Each comment is assigned a class according to the following ranges: [80, 100]\% 
for \dir, (20,80)\% for \rel, and [0,20]\% for \irel. Threshold values manually 
tuned on the training data.


As for the contrastive submission~2, we built a binary classifier \dir vs. 
\texttt{NO-}\dir based on logistic regression. The comments are then sorted 
according to the classifier's prediction confidence and the final labels are 
assigned accordingly: \dir for the 1st ranked, \rel for the 2nd ranked, and 
\irel for the rest. Only lexical similarities are included as features 
(discarding those weighted with idf variants).


The performance of these two submissions is below but close to that of the 
primary one (F$_1=76.60$ and $76.97$), particularly when identifying \irel 
comments. 

\paragraph{English A}

For our contrastive submission~1, the same machine learning schema as for the 
primary submission is used, but now using 
SVM$^\mathrm{light}$~\cite{Joachims:99}. This toolkit allows us to deal with 
the class imbalance by tuning the $j$ parameter (cost of making mistakes on 
positive examples). This time the $C$ hyper-parameter is set to the default 
value. As we focused on improving the performance on \pot instances, we 
obtained better results for this category (F$_1=17.44$), surpassing the overall 
performance from the primary submission (F$_1=56.40$).

Our contrastive submission~2 operates in the same way as the Arabic contrastive 
submission~1. The applied ranges are the same, but this time for \good, \pot, 
and \bad. Some heuristics override the so generated decisions: $c$ is classified 
as \good if it includes a URL, starts with an imperative verb (\eg \textit{try}, 
\textit{view}, \textit{contact}, \textit{check}), or contains \textit{yes words} 
(\eg \textit{yes}, \textit{yep}, \textit{yup}) or \textit{no words} (\eg 
\textit{no}, \textit{nooo}, \textit{nope}). Comments written by the author of 
the question or including acknowledgments are considered dialogues and 
classified as \bad. The result of this submission is slightly lower than the 
others' (F$_1=51.97$), where the automatic learning allows for better 
predictions.

\paragraph{English B}

Our contrastive submission~1 is identical to our primary one, but it uses both 
the training and the development data for training the model. The reason behind 
the disastrous results (F$_1=25.23$) is a buggy implementation of some of the 
polarity features (\cf Section~\ref{sub:polarity}) and the lack of statistics 
for properly estimating category-level user profiles (\cf 
Section~\ref{sub:profile}). 
 
 
The contrastive submission~2 consists of a rule-based system. A comment is 
labeled as \yes if it starts with affirmative words: \textit{yes}, 
\textit{yep}, \textit{yeah}, etc. It is labeled as \no if it starts with 
negative words: \textit{no}, \textit{nop}, \textit{nope}, etc. The answer to $q$ 
becomes that of the majority of the comments ---\unsure in case of tie. It is 
worth noting the comparably high performance when dealing with \unsure questions 
(F$_1=47.06$) with this simple rationale.
\section{Pots-Submission Experiments}
\label{sec:discussion}


\subsection{Arabic} \label{sec:discussionArabic}

Table~\ref{tab:aftertaskarabic} shows the results obained after discarding each 
of the different features families.

\begin{table}%[h]	
\begin{tabular}{|l|crrr|}
% \hline
%   \multicolumn{5}{c}{Arabic} \\
\hline  
 Subm. without& \bf \texttt{DIR} & \bf \texttt{REL} & \bf \texttt{IREL} & 
\bf \texttt{MACRO} \\\hline
 $n$-grams	&	&	&	& 76.75	\\
 cont$_1$	&	&	&	& 67.74	\\
 no max iyas	&	&	&	& 79.13	\\
 no max sim	&	&	&	& 78.37	\\
 no max iyas sim&	&	&	& 78.69	\\ 
  \hline
 \end{tabular}
 \caption{Post-competition experiments Arabic\label{tab:aftertaskarabic}}
 \end{table}

\subsection{English Task A} \label{sec:discussiona}

Table~\ref{tab:aftertaska} shows the results obtained on the test set by 
considering both the different subsets of features in isolation (\textit{only 
with}) or all the features except for a subset (\textit{without}). According to 
these figures, the heuristic features seem to be the most useful, followed by 
the context-based information. The best performing subsets are close to that 
combining all the features (\cf Table~{tab:results}).


\begin{table}%[h]	
\begin{tabular}{|l|cccc|}
% \hline
%   \multicolumn{5}{c}{Subtask A} \\
\hline  
 Only with 	& \bf \good & \bf \bad & \bf \texttt{POT} & \bf \texttt{MACRO} 
\\\hline
context	&	&	&	& 47.90	\\
 $n$-grams	&	&	&	& 44.86\\
 heuristics	&	&	&	& 52.57\\
 Similarities	& 	&	&	& 46.16	\\
 \,\,\,\,\, lexical	&	&	&	& 44.82	\\
 \,\,\,\,\, syntactic&	&	&	& 36.47	 \\
 \,\,\,\,\, semantic&	&	&	& 42.16	 \\\hline
 Without 	& \bf \good & \bf \bad & \bf \texttt{POT} & \bf 
\texttt{MACRO} 
\\\hline
 context	&	&	&	& 51.49	\\
 $n$-grams	&	&	&	& 55.17\\
 heuristics	&	&	&	& 48.60\\
 Similarities	& 	&	&	& \blue{??.??}	\\
 \,\,\,\,\, lexical&	&	&	& 53.34	\\
 \,\,\,\,\, syntactic&	&	&	& 53.73	 \\
 \,\,\,\,\, semantic&	&	&	& 53.50	 \\ 
  
  \hline
 \end{tabular}
 \caption{Post-competition experiments English A \label{tab:aftertaska}}
 \end{table}



\subsection{English Task B} \label{sec:discussionb}

After the submission we investigated on the reasons why learning on the training 
only was considerably better than learning on the union of the training and 
development sets. 
The sequences of predicted target labels on the test set in the two learning 
scenarios showed considerable differences: when learning on the union of the 
training and development sets the predicted labels were YES on all but three 
cases. 
After correcting a bug, the results obtained by learning on the union of the 
training and development sets were the ones in the ``after$_1$'' submission in 
the first row of Table~\ref{tab:aftertaskb}, i.e.  
a Macro F1 value of $51.98$. Learning on the training set only still gives a 
higher macro F1 of $69.35$, but the sequences of predicted labels are now more 
consistent and the difference might not be significant (REFERENCE TO TASK 
DESCRIPTION PAPER). 
We observed that the values of those features counting the number of Good, Bad, 
and Potential comments within categories from the same user (\cf 
Section~\ref{sub:app_enB}) vary greatly when computed on the training or 
training+dev datasets. 
This is due to the fact that the number of comments of a user for a category is, 
in most cases, too limited to generate reliable statistics. 
After discarding these three features, the obtained Macro F1 value is $55.95$ 
(see ``after$_2$'' submission in Table~\ref{tab:aftertaskb}), which represents a 
higher performance than the ones obtained during at submission time.

\begin{table}%[h]	
\begin{tabular}{|l|cccc|}
% \hline
%   \multicolumn{5}{c}{Subtask B} \\
\hline  
 Subm.		& \bf \yes & \bf \no & \bf \unsure & \bf \texttt{MACRO}	 \\
  \hline
  \,\,after$_1$	& $78.79$	& $57.14$	& $20.00$	& $51.98$ \\
  \,\,after$_2$ & $85.71$	& $57.14$	& $25.00$ 	& $55.95$ \\
  \hline
 \end{tabular}
 \caption{\label{tab:aftertaskb}}
 \end{table}


\section{Conclusions and Future Work}
\label{sec:conclusion}

We have presented the system developed by the team of the Qatar Computing Research Institute (QCRI)
for participating in SemEval-2015 Task 3 on Answer Selection in Community Question Answering.
We used a supervised machine learning approach and a manifold of features including word
$n$-grams, text similarity, sentiment dictionaries, the presence of specific words, the context of a comment, some heuristics, etc.
Our approach was the best performing one in the Arabic task,
and the third best in the two English tasks.

We further presented a detailed study of which kinds of features helped most for each language and for each subtask, which should help researchers focus their efforts in the future.

In future work, we plan to use richer linguistic annotations, more complex kernels,
and large semantic resources.% such as DBpedia and linked open data.


\begin{footnotesize}
\section*{Acknowledgments}
This research is developed by the Arabic Language Technologies (ALT) group at the Qatar Computing Research Institute (QCRI), Qatar Foundation in collaboration with MIT. It is part of the Interactive sYstems for Answer Search (Iyas) project.
\end{footnotesize}

% \begin{thebibliography}{}
\bibliographystyle{naaclhlt2015}
\bibliography{semeval15}


% \bibitem[\protect\citename{Aho and Ullman}1972]{Aho:72}
% Alfred~V. Aho and Jeffrey~D. Ullman.
% \newblock 1972.
% \newblock {\em The Theory of Parsing, Translation and Compiling}, volume~1.
% \newblock Prentice-{Hall}, Englewood Cliffs, NJ.
% 
% \bibitem[\protect\citename{{American Psychological Association}}1983]{APA:83}
% {American Psychological Association}.
% \newblock 1983.
% \newblock {\em Publications Manual}.
% \newblock American Psychological Association, Washington, DC.
% 
% \bibitem[\protect\citename{{Association for Computing Machinery}}1983]{ACM:83}
% {Association for Computing Machinery}.
% \newblock 1983.
% \newblock {\em Computing Reviews}, 24(11):503--512.
% 
% \bibitem[\protect\citename{Chandra \bgroup et al.\egroup }1981]{Chandra:81}
% Ashok~K. Chandra, Dexter~C. Kozen, and Larry~J. Stockmeyer.
% \newblock 1981.
% \newblock Alternation.
% \newblock {\em Journal of the Association for Computing Machinery},
%   28(1):114--133.
% 
% \bibitem[\protect\citename{Gusfield}1997]{Gusfield:97}
% Dan Gusfield.
% \newblock 1997.
% \newblock {\em Algorithms on Strings, Trees and Sequences}.
% \newblock Cambridge University Press, Cambridge, UK.

% \end{thebibliography}

\end{document}
