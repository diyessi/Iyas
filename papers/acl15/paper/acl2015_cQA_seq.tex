%
% File acl2015.tex
%
% Contact: car@ir.hit.edu.cn, gdzhou@suda.edu.cn
%%
%% Based on the style files for ACL-2014, which were, in turn,
%% Based on the style files for ACL-2013, which were, in turn,
%% Based on the style files for ACL-2012, which were, in turn,
%% based on the style files for ACL-2011, which were, in turn, 
%% based on the style files for ACL-2010, which were, in turn, 
%% based on the style files for ACL-IJCNLP-2009, which were, in turn,
%% based on the style files for EACL-2009 and IJCNLP-2008...

%% Based on the style files for EACL 2006 by 
%%e.agirre@ehu.es or Sergi.Balari@uab.es
%% and that of ACL 08 by Joakim Nivre and Noah Smith

\documentclass[11pt]{article}
\usepackage{acl2015}
\usepackage{times}
\usepackage{url}
\usepackage{latexsym}

%\setlength\titlebox{5cm}

% You can expand the titlebox if you need extra space
% to show all the authors. Please do not make the titlebox
% smaller than 5cm (the original size); we will check this
% in the camera-ready version and ask you to change it back.


\title{Neighbor Answers Matter: Sequence Information Gives a Hand to Answer Selection in Community Question Answering}

\author{First Author \\
  Affiliation / Address line 1 \\
  Affiliation / Address line 2 \\
  Affiliation / Address line 3 \\
  {\tt email@domain} \\\And
  Second Author \\
  Affiliation / Address line 1 \\
  Affiliation / Address line 2 \\
  Affiliation / Address line 3 \\
  {\tt email@domain} \\}

\date{}

\begin{document}
\maketitle
\begin{abstract}
Community question answering fora allow people to get answers to their questions from a wide on-line community. The diversity and openness of these communities allow for quick comments interaction which includes both accurate and inaccurate answers to the questions. In this paper we explore the automatic detection of good answers within a thread of comments to a given questions. Our experiments, comparable to the state of the art in the topic, show that considering the relationships among the comments of the whole thread ---either in the form of features or in the machine learning model--- allows for improving the results by blablabla
\end{abstract}

\input{acl2015_cQA_seq_intro}

\section{Experiments}

\begin{table*}
\caption{Multiclass setting. All includes all the features, but $n$-grams; sub includes all the features available in the pipeline, except for LSA; noc is like sub, without context features). }

\begin{tabular}{lccccccccccccc}
\hline
				& \multicolumn{3}{c}{Good}	& \multicolumn{3}{c}{Potential} &\multicolumn{3}{c}{Bad} & \multicolumn{4}{c}{Overall}\\
						&	P			& R			& F			&	P			& R			& F			& 	P			& R			& F			& 	P				& R			& F			& A	\\
\hline
SVM$_{all}$		& 	74.59	& 73.02	& 73.80	& 15.62	& 17.96	& 16.71	& 	71.78	& 71.43	& 71.60	& 	54.00	& 54.14	&	54.04	& 67.71	\\
SVM$_{sub}$	& 73.18	& 75.83	& 74.48	& 13.41	& 13.17	& 13.29	& 73.17	& 	70.20	& 	71.65	& 	53.26	& 53.07	&	53.14	& 68.22	\\
SVM$_{noc}$	& 	73.33	& 67.30	& 	70.19	&  6.56	& 	7.19		&  6.86	& 64.69	& 	69.65	& 67.22	& 48.19	& 48.15	&	48.09	& 63.31	\\
%				& 		& 	& 	& 	& 	& 	& 		& 	& 	& 		& 	&		& 	\\
%			& 		& 	& 	& 	& 	& 	& 		& 	& 	& 		& 	&		& 	\\
%				& 		& 	& 	& 	& 	& 	& 		& 	& 	& 		& 	&		& 	\\
				
\hline
\end{tabular}
\end{table*}

\begin{table*}
\caption{Binary setting. All includes all the features, but $n$-grams; sub includes all the features available in the pipeline, except for LSA; noc is like sub, without context features). }

\begin{tabular}{lccccccccccccc}
\hline
				& \multicolumn{3}{c}{Good}	& \multicolumn{3}{c}{Potential} &\multicolumn{3}{c}{Bad} & \multicolumn{4}{c}{Overall}\\
						&	P			& R			& F			&	P			& R			& F			& 	P			& R			& F			& 	P				& R			& F			& A	\\
\hline
SVM$_{all}$		& 		& 	& 	& 	& 	& 	& 		& 	& 	& 		& 	&		& 	\\
SVM$_{sub}$	& 		& 	& 	& 	& 	& 	& 		& 	& 	& 		& 	&		& 	\\
SVM$_{noc}$	& 		& 	& 	& 	& 	& 	& 		& 	& 	& 		& 	&		& 	\\
CRF$_{all}$		& 		& 	& 	& 	& 	& 	& 		& 	& 	& 		& 	&		& 	\\
CRF$_{sub}$	& 		& 	& 	& 	& 	& 	& 		& 	& 	& 		& 	&		& 	\\
LogReg$_{all}$	& 		& 	& 	& 	& 	& 	& 		& 	& 	& 		& 	&		& 	\\
LogReg$_{sub}$	& 		& 	& 	& 	& 	& 	& 		& 	& 	& 		& 	&		& 	\\				
\hline
\end{tabular}
\end{table*}
%{\bf Citations}: Citations within the text appear in parentheses
%as~\cite{Gusfield:97} or, if the author's name appears in the text
%itself, as Gusfield~\shortcite{Gusfield:97}.  Append lowercase letters
%to the year in cases of ambiguity.  Treat double authors as
%in~\cite{Aho:72}, but write as in~\cite{Chandra:81} when more than two
%authors are involved. Collapse multiple citations as
%in~\cite{Gusfield:97,Aho:72}. Also refrain from using full citations
%as sentence constituents. We suggest that instead of
%\begin{quote}
%  ``\cite{Gusfield:97} showed that ...''
%\end{quote}
%you use
%\begin{quote}
%``Gusfield \shortcite{Gusfield:97}   showed that ...''
%\end{quote}




\section*{Acknowledgments}


% include your own bib file like this:
%\bibliographystyle{acl}
%\bibliography{acl2015}

\begin{thebibliography}{}

\bibitem[\protect\citename{Aho and Ullman}1972]{Aho:72}
Alfred~V. Aho and Jeffrey~D. Ullman.
\newblock 1972.
\newblock {\em The Theory of Parsing, Translation and Compiling}, volume~1.
\newblock Prentice-{Hall}, Englewood Cliffs, NJ.

\bibitem[\protect\citename{{American Psychological Association}}1983]{APA:83}
{American Psychological Association}.
\newblock 1983.
\newblock {\em Publications Manual}.
\newblock American Psychological Association, Washington, DC.

\bibitem[\protect\citename{{Association for Computing Machinery}}1983]{ACM:83}
{Association for Computing Machinery}.
\newblock 1983.
\newblock {\em Computing Reviews}, 24(11):503--512.

\bibitem[\protect\citename{Chandra \bgroup et al.\egroup }1981]{Chandra:81}
Ashok~K. Chandra, Dexter~C. Kozen, and Larry~J. Stockmeyer.
\newblock 1981.
\newblock Alternation.
\newblock {\em Journal of the Association for Computing Machinery},
  28(1):114--133.

\bibitem[\protect\citename{Gusfield}1997]{Gusfield:97}
Dan Gusfield.
\newblock 1997.
\newblock {\em Algorithms on Strings, Trees and Sequences}.
\newblock Cambridge University Press, Cambridge, UK.

\end{thebibliography}

\end{document}
